\documentclass[10pt]{extarticle}

%Paquetes utilizados en esta tarea
\usepackage{fullpage}
\usepackage[utf8]{inputenc}
\usepackage[spanish]{babel}
\usepackage{epsfig}
\usepackage{amsmath}
\usepackage{amssymb}
\usepackage{epstopdf}
\usepackage[hidelinks]{hyperref}
\usepackage{xcolor}
\usepackage{algorithmic}
\usepackage[nothing]{algorithm}

%Definiciones de comandos, para reutilizar secuencias frecuentes o ahorrar
%código
\newcommand{\RR}{\mathbb{R}}
\newcommand{\lb}{\\~\\}
\newcommand{\la}{\leftarrow}

\newcommand{\twopartdef}[4]
{
	\left\{
		\begin{array}{ll}
			#1 &  \text{si }#2 \\
			#3 &  \text{si }#4
		\end{array}
	\right.
}

\newcommand{\threepartdef}[6]
{
	\left\{
		\begin{array}{ll}
			#1 &  \text{si }#2 \\
			#3 &  \text{si }#4 \\
			#5 &  \text{si }#6
		\end{array}
	\right.
}

\makeatletter

\makeatother

\begin{document}

\begin{tabular}{ccl}
 \begin{tabular}{c}
 \includegraphics[width=2.5cm]{imgs/logo.pdf}
\end{tabular}
&\ \ \ &
\begin{tabular}{l}
 PONTIFICIA UNIVERSIDAD CATÓLICA DE CHILE               \\
 DEPARTAMENTO DE CIENCIA DE LA COMPUTACIÓN              \\
 IIC3524 {-} Tópicos avanzados de sistemas distribuidos \\
\end{tabular}
\end{tabular}

\begin{center}
 \bf {\Huge Tarea 2}

 \vspace{0.2cm}
 \bf 27 de junio de 2017

 \vspace{0.2cm}
 \bf Nicolás Gebauer {-} 13634941

 \vspace{0.2cm}
 \bf \href{https://github.com/negebauer}{\color{blue!60} @negebauer} {-} \href{https://github.com/negebauer/IIC3524-T2}{\color{blue!60}repo T2}
 \noindent\rule{16cm}{0.05pt}

 \vspace{0.5cm}
 % \bf {\huge Análisis}
\end{center}

\subsection*{Análisis de rendimiento}
Se ejecutó el programa en su versión secuencial y paralela. La versión paralela se corrió con 4 procesos en cada nodo. Los nodos utilizados fueron 3, tripio, trauco y caleuche.\\
Los tiempos de procesamiento para encontrar el camino más corto, junto con los speedups respectivos, se presentan a continuación.\\
Las tablas describen para cada test el tiempo que tomo ejecutando \textit{time} en los campos \textit{real} y \textit{user}. S son los tiempos secuenciales, P los paralelos.
Los tamaños de las pruebas son
\begin{align*}
 t3: & 12 \\
 t4: & 16 \\
 t5: & 17 \\
 t6: & 18 \\
\end{align*}
\begin{center}
 \includegraphics[width=6cm]{imgs/table_seconds.png}\\
 \footnotesize{Tiempos de ejecución en segundos}\\
\end{center}

\begin{center}
 \includegraphics[width=6cm]{imgs/table_minutes.png}\\
 \footnotesize{Tiempos de ejecución en minutos}\\
\end{center}

\begin{center}
 \includegraphics[width=6cm]{imgs/table_speedup.png}\\
 \footnotesize{Speedups}\\
\end{center}

Se grafican las tablas de tiempo y speedup a continuación

\begin{center}
 \includegraphics[width=12cm]{imgs/graph_seconds.png}\\
 \footnotesize{Tiempos de ejecución en segundos}\\
\end{center}

\begin{center}
 \includegraphics[width=12cm]{imgs/graph_speedup.png}\\
 \footnotesize{Speedups}\\
\end{center}

\subsection*{Discusión}
Se aprecia que para un $N$ pequeño ($t3$ por ejemplo) el speedup es inferior a $1$, es decir, el programa toma más tiempo en la versión paralela. Esto se debe a que el problema es demasiado pequeño como para aprovechar un cómputo paralelo. El preparar los datos en cada nodo junto con la comunicación entre ellos toma más tiempo que simplemente resolver el problema en un local.\\
Al aumentar $N$ el speedup se hace mucho más importante, llegando a un $505\%$ para el tiempo real en $t6$.\\

\end{document}
